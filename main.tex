\documentclass[a4paper,12pt,titlepage]{extarticle}

\usepackage[T2A]{fontenc}
\usepackage[utf8x]{inputenc}
\usepackage[russian]{babel}
\usepackage{indentfirst}
\usepackage{graphicx}
\usepackage[left=2cm,top=2cm,right=1.7cm,bottom=2cm,nohead]{geometry}
\usepackage[font={small,it}]{caption}
\usepackage{wrapfig}
\usepackage[colorlinks=false]{hyperref}
\usepackage{draftwatermark}
\SetWatermarkFontSize{36pt}
\SetWatermarkScale{4}

\usepackage{enumitem}
\setlist{nolistsep}
%\setlist[1]{\labelindent=\parindent}
%\setlist[itemize]{leftmargin=*}
\setlist[itemize,1]{label=-}

\author{ОО <<Минское велосипедное общество>>}
\date{Минск, 2012}
\title{Маркировка туристских маршрутов. Методические рекомендации.}

\sloppy
\begin{document}

\maketitle

\section{Введение.}

Туристская маркировка --- это система специальных условных обозначений, которые наносятся на различные предметы или устанавливаются на местности для разметки рекомендуемых маршрутов туристских путешествий, походов и прогулок.

Она включает различные виды марок, направляющих стрелок, указателей маршрутов и иных знаков, которые в своих символах,
форме, цвете и буквенно-цифровом коде несут необходимую для туриста путевую информацию о направлениях и расстояниях до
объектов туристско-экскурсионного осмотра и обслуживания, о местах, предназначенных для привалов, естественных
препятствиях и опасных участках, о рекомендуемых или запрещаемых формах поведения туристов на маршруте.\cite{metodussr}

Цель маркировки --- помочь туристу, экскурсанту, отдыхающему наилучшим образом и с наименьшими затруднениями совершить
намеченный поход, экскурсию или прогулку.

Маркировка предоставляет информацию, необходимую для ориентирования на местности и передвижения по маршруту. Хорошая
маркировка позволяет неподготовленному человеку пройти весь маршрут, не используя никакие дополнительные средства
ориентирования.

В настоящее время в Беларуси разработан и введён в действие государственный стандарт СТБ~1821--2007 «Знаки
информационные туристические. Общие технические условия.» Однако данный стандарт не затрагивает вопросы маршрутного
ориентирования туристов, а лишь определяет способы информирования о различных объектах.

Данные технические рекомендации разработаны с целью облегчить создание привлекательных и удобных для посещения
туристических маршрутов вне населённых пунктов.

\section{Требования, предъявляемые к маркировке.}

Система маркировки туристских маршрутов должна отвечать следующим требованиям:
\begin{description}
	\item[Информационная насыщенность] позволяет путешественнику уверенно ориентироваться на местности, следовать по
		выбранному маршруту, распознавать потенциально интересующие его места на маршруте (место для привала,
		достопримечательность, видовая точка и т.д.);
	\item[Наглядность.] Символика и принципы маркировки маршрута должны быть интуитивно понятны и очевидны любому
		человеку, попавшему на маршрут, в том числе не имеющему туристского опыта.
	\item[Единообразие.] На протяжении всего маршрута, других маршрутов в данной местности, вид маркировки и
		принципы её размещения должны быть постоянными --- одинаковые элементы маршрутов должны маркироваться
		одинаково.
	\item[Эстетичность.] Маркировка не должна вызывать чувство чужеродности и негармоничности с окружающей её
		местностью, но в то же время быть хорошо заметной и узнаваемой.
	\item[Тактичность.] Количество запрещающих и предписывающих указаний, выраженных с помощью условных знаков,
		должно быть сведено к минимуму. Предпочтительнее использовать более <<мягкие>> формы ---
		предупреждающие, информирующие.
	\item[Логичность.] Маркировка должна быть естественным <<помощником>> при прохождении маршрута, следовать
		принципу наименьшего удивления.
	\item[Открытость.] Маркированный маршрут не может существовать как <<вещь-в-себе>>, он является частью окружающей
		его местности, и любой путешественник вправе выбрать свой собственный вариант прохождения. Маркировка не
		должна создавать двусмысленностей при попадании на маршрут в его середине, должна позволять
		беспрепятственно отклониться от маршрута, вернуться на него, перейти на другой маршрут.
	\item[Технологичность.] Одно из наиболее важных требований, поскольку, в отличие от информационных знаков,
		маршрутные наносятся часто и на самые различные поверхности: от бетона до древесной коры.
		Технология должна обеспечивать простое и быстрое нанесение знака на любую поверхность, удержание его на
		этой поверхности в течение всего расчётного срока эксплуатации. По той же причине маркировка должна
		быть как можно более простой с графической точки зрения.
	\item[Экономичность.] Связана с технологичностью. Низкая стоимость маркировки должна обеспечиваться простотой
		создания отдельных знаков, длительным сроком их службы.
	\item[Безопасность для автоводителей.] Маркировка туристических маршрутов не должна ни при каких условиях быть
		принятой за какой-либо существующий дорожный знак или иную метку, обеспечивающую безопасность движения
		автомобилей по дорогам.
\end{description}

\section{Планирование маршрута.}

Вопросы детального планирования маршрута выходят за рамки данного документа. Здесь приведены лишь общие принципы:
\begin{itemize}
	\item Маршрут должен разрабатываться и эксплуатироваться с помощью местных сообществ, в этом заинтересованных:
		туристические клубы, владельцы агроэкоусадеб, краеведческие организации, музеи и т.д.
	\item Маршруты должны быть рассчитаны на дальнейшее развитие туристической сети, соединение их с другими,
		включение в глобальные туристические маршруты, такие, как EuroVelo.
	\item Для каждого маршрута должен быть определён уровень его сложности и все элементы маршрута должны ему
		соответствовать. Например, не следует включать брод через реку в маршрут, предназначенный для семейного
		отдыха.
	\item Желательно, чтобы маршрут был пригоден для прохождения как в прямом, так и в обратном направлениях.
	\item В ходе планирования следует избегать дорог общего пользования с интенсивным либо скоростным автомобильным
		движением.
\end{itemize}


\section{Маршрутная марка.}

Маршрутная марка является основным элементом маркировки на маршруте. Она представляет собой знак, символизирующий
принадлежность конкретной точки местности к определенному туристскому маршруту. Цель марки --- поддерживать у туриста
уверенность в правильном движении по маршруту, который подсказывается обычно направлением дороги, тропы, лыжни, водного
потока.

При отсутствии на местности четкой тропиночной сети, необходимых ориентиров, а также в тех районах, где изменение погоды
резко снижает видимость и затрудняет ориентировку, маршрутные марки могут выполнять роль основного, а иногда и
единственного, направляющего ориентира.

\subsection{Формат марки.}

\begin{figure}
	\centering
	\includegraphics[width=0.7\textwidth]{image-mark}
	\caption{Маршрутная марка}\label{fig:mark}
\end{figure}

Маршрутная марка имеет формат прямоугольника, на белом поле которого нанесена горизонтальная цветная полоса высотой в
треть высоты марки (рис.~\ref{fig:mark}). В зависимости от характера рельефа, условий видимости, практикуемых в данной
местности видов туризма и других особенностей устанавливается три формата маршрутных марок:
\begin{itemize}
	\item Нормальный (основной) формат --- $90\times150$~мм. Ширина цветной горизонтальной полосы --- 30~мм.
	\item Увеличенный формат --- $180\times300$~мм. Применяется в условиях плохой видимости, удалённости объекта с маркой от
тропы, сильной неровности поверхности, на которую наносится. Также может использоваться в случаях, если требуется
нанести на марку дополнительные обозначения (номер маршрута, тип транспортного средства и т~д.).
\item Уменьшенный формат --- $60\times100$~мм. Может применяться в эстетических соображениях, на объектах небольшого размера с
гладкой поверхностью (например, опора дорожного знака), в качестве вспомогательной маркировки.
\end{itemize}

Допустимы и иные размеры знаков при сохранении пропорций и следовании принципу единообразия.

\begin{wrapfigure}{r}{0.27\textwidth}
%\begin{figure}[ht]
	\centering
	\includegraphics[height=0.25\textheight]{image-mark-stacked}
	\caption{Комбинированная маршрутная марка}\label{fig:mark-stacked}
%\end{figure}
\end{wrapfigure}

При обозначении нескольких маршрутов на одном объекте допускается располагать марки <<внахлёст>>, совмещаяя белые полосы
соседних марок. Такая комбинированная марка выглядит как прямоугольник с чередующимися белыми и цветными полосами равной
ширины (рис.~\ref{fig:mark-stacked}).

\subsection{Цвет.}
Цвет горизонтальной полосы выбирается из ограниченного количества чистых тонов исходя из соображений лёгкой
идентификации маршрута среди других маршрутов, проходящих в данной местности.

В \cite{metodussr} предложено с помощью цвета обозначать сложность маршрута. Подобная классификация просматривается и в
СТБ~1821--2007, где на соответствующих знаках трудность кодируется цветной полосой  (лёгкий --- зелёный, умеренный ---
жёлтый, трудный --- красный, опасный --- чёрный). Учитывая, что в условиях Беларуси в одной местности редко присутствуют
маршруты разной сложности, но может пересекаться большое количество маршрутов одинаковой сложности, использование такого
кодирования в маршрутных марках нецелесообразно.

Для маршрутных марок рекомендуются следующие цвета (рис.~\ref{fig:mark-colors},~а):
\begin{itemize}
	\item красный (предпочтительный для веломаршрутов);
	\item зелёный (предпочтительный для пешеходных маршрутов);
	\item синий (предпочтительный для водных маршрутов);
	\item жёлтый.
\end{itemize}

В случае, если возникает необходимость в маркировке большого числа маршрутов на ограниченной территории, то в целях
предупреждения пересечения одинаково маркированных трасс можно использовать на марках комбинацию из двух цветов:
красного и синего, красного и желтого, зеленого и желтого (рис.~\ref{fig:mark-colors},~б). От сочетания синего и зеленого, а при возможности
также красного и зеленого, синего и желтого цветов следует воздерживаться.

\begin{figure}[ht]
	\centering
	\includegraphics[width=0.7\textwidth]{image-mark-colors}
	\caption{Цвета маршрутных марок}\label{fig:mark-colors}
\end{figure}


При маркировке радиальных ответвлений от основного маршрута (радиальное ответвление кончается тупиком) или вариантов
маршрута (например, тропа на каком-то участке уходит в сторону от основного маршрута, а затем возвращается к нему)
рекомендуется использовать специальную марку с прямоугольником такого же цвета, как и на основном маршруте (рис.~\ref{fig:mark-colors},~в).

При размещении на светлой поверхности для улучшения видимости можно обвести марку каймой тёмного цвета.

\subsection{Размещение марок.}

Одна из главных задач маркировки --- разместить маршрутные марки так, чтобы они были заметны для туристов с как можно
большего расстояния. В зависимости от конкретных условий марки могут быть установлены на различных местных предметах как
естественных, так и искусственных.

На загородных территориях в лесной зоне, в парках,
лесопарках и зонах массового отдыха марки, как правило, размещаются на деревьях, растущих вдоль дороги или тропы, по
которой проходит туристский маршрут. Для нанесения знака рекомендуется выбирать наиболее заметные, выделяющиеся среди
других деревья, желательно с гладкой корой и толщиной ствола на уровне глаз человека 150--250~мм. Деревья не должны быть
поражены болезнями или лесными вредителями и не должны иметь слишком низкорасположенной кроны, закрывающей ствол от глаз
человека. Не следует ставить марки на хвойные деревья с обильным течением смолы и на деревья, отмеченные лесничеством
для вырубки.

В населенных пунктах, вблизи сельскохозяйственных угодий, вдоль полевых дорог и троп маршрутные марки можно ставить
(получив на это соответствующее разрешение) на углы зданий и сооружений, межевые и квартальные столбы, заборы, ограды и
т.~п. При выборе мест для размещения марок следует, помимо прочего, обращать внимание на долговечность опоры и ее
эстетичность: нельзя ставить маркировку на временные и подлежащие сносу сооружения, на ржавые металлические опоры, на
предметы, пропитанные битумом, смолой, льняными маслами.

Частота установки маршрутных марок зависит от особенностей местности, густоты дорожной сети и вида туристского маршрута.

\begin{figure}
	\centering
	\includegraphics[width=\textwidth]{image-mark-placement}
	\caption{Размещение маршрутных марок}\label{fig:mark-placement}
\end{figure}

На тех участках пешеходного пути, где имеется одна явно выраженная, не имеющая перекрестков или ответвлений тропа или
дорога, марки устанавливаются через 200--250~м. При пересечении основной тропы (дороги), по которой идет маршрут,
второстепенными тропами марки следует ставить в 10--20~м до и после такого пересечения (рис.~\ref{fig:mark-placement},~а,~б).

При наличии на местности нескольких параллельных троп, троп, расходящихся под острыми углами, при движении туристов по
лесу вне дорожной сети марки следует размещать так, чтобы от каждого знака было видно предыдущий и последующий знаки.

Обязательна установка марки (причем на заметном и видном месте с обоих направлений маршрута) на всех поворотах маршрута.
При особо резких изменениях направления, на развилках троп, при сходе маршрута с хорошей дороги на слабо заметную тропу,
а также в других местах, где у туристов может возникнуть сомнение в правильности избранного пути, перед подобным
поворотом (и сразу после него) устанавливаются две-три марки с интервалами в 10--20~м между ними (рис.~\ref{fig:mark-placement},~в).

Такое <<сгущение>> знаков необходимо для повышения внимания путешественников на сложном участке, для гарантии правильности
направления их движения и тогда, когда один-два знака останутся незамеченными или окажутся по какой-либо причине
уничтоженными.

На обширных открытых участках местности без постоянной тропиночной сети частота размещения марок на турах и
маркировочных столбах определяется их видимостью при неблагоприятных метеоусловиях (дожде, тумане, мгле) и составляет
25--50~м.

Рекомендуемая высота размещения маршрутных марок над землей --- на уровне глаз человека (на пеших маршрутах около 160~см).
Если высота опор не позволяет этого и марки помещаются ниже, например, наносятся на камни, интервалы между знаками
следует сокращать,

Марки по возможности надо размещать по одну сторону от тропы или дороги. Переход знаков с одной стороны тропы на другую
может сигнализировать о близком изменении направления туристского маршрута --- о повороте в ту сторону, на которую
<<перешла>> маркировка. Место поворота маршрута может быть отмечено маркировкой сразу с двух сторон тропы (<<ворота>>) (рис.~\ref{fig:mark-placement},~г).

Марки на деревьях или столбах размещают с учетом их толщины. При диаметре опоры 100--200~мм марку следует ставить на той
стороне ствола, которая обращена к тропе, что обеспечивает видимость знака по обоим направлениям маршрута (внутренняя
маркировка).

На более толстых деревьях (при использовании маршрута в двух направлениях) марки приходится ставить с обеих сторон по
ходу возможного движения туристов (двухсторонняя маркировка).

На стволах тонких деревьев, столбах с диаметром менее 100~мм необходимо делать маркировку в виде сплошной ленты
(кольцевая маркировка).

При совмещении на одной тропе, дороге или одном направлении двух и более разных маркировок нельзя ставить на местные
предметы марки поочередно, попеременно. Такие участки должны размечаться с помощью комбинированных марок. При их
установке следует учитывать, что размеры марок по высоте могут значительно превышать стандартные форматы, и поэтому
требуют более тщательного выбора опор.

Для облегчения обнаружения маршрутной марки в тёмное время суток можно разместить поблизости от неё метку из световозвращающей самоклеящейся
плёнки. При выборе цвета и формата такой метки следует учитывать вопросы безопасности движения автомобилей. Размещение
меток не должно приводить к неправильной оценке расстояния до стоящих у дороги деревьев, опор
(метка должна располагаться на ближайших к дороге объектах).

\subsection{Направляющая стрелка}
\begin{figure}
	\centering
	\includegraphics[width=0.7\textwidth]{image-arrow}
	\caption{Направляющая стрелка}\label{fig:arrow}
\end{figure}

Направляющая стрелка --- вспомогательный знак. Он сопутствует маршрутным маркам, когда объем информации, сообщаемой
марками, недостаточен для уверенного движения туристов (рис.~\ref{fig:arrow}).

Направляющие стрелки ставят на резких поворотах маршрута, на развилках, при выходе маршрута на открытые участки.
Возможно применение нескольких видов стрелок.

\begin{figure}
	\centering
	\includegraphics[width=0.7\textwidth]{image-arrows}
	\caption{Виды направляющих стрелок}\label{fig:arrows}
\end{figure}

Прямая стрелка предлагает продолжать движение прямо в указываемом направлении (рис.~\ref{fig:arrows},~а).

Двухсторонняя стрелка указывает на радиальное ответвление маршрута, оканчивающееся тупиком, предлагает
совершить экскурсию в сторону от основной трассы путешествия и вернуться затем на исходную точку. Направление
радиального ответвления обозначается цветной полосой в правой (левой) части знака (рис.~\ref{fig:arrows},~б).

\begin{figure}[ht]
	\centering
	\includegraphics[height=0.25\textheight]{image-arrow-stacked}
	\caption{Комбинированная марка с направляющими стрелками}\label{fig:arrow-stacked}
\end{figure}


\subsection{Информационные знаки}
Кроме маршрутных марок, на маршруте могут устанавливаться дополнительные виды знаков: информационные щиты со схемой и
описанием маршрутов, указатели направлений и расстояний, знаки начала и конца маршрута, иные информационные знаки.

Удобными точками для их размещения являются места пересечения нескольких маршрутов, узловые и поворотные пункты
туристских трасс, обзорные точки, места сосредоточения экскурсантов у музеев и памятников, места туристских стоянок.
Наиболее целесообразно размещать такие виды информации на охраняемых территориях: у туристских приютов, объектов
экскурсионного осмотра, перед заповедниками, лесными кордонами и т.~п.

В ходе разработки и установки информационных знаков следует руководствоваться требованиями СТБ~1821--2007. В случае
отсутствия в стандарте требуемых знаков, проводится их индивидуальное проектирование с использованием принципов,
типоразмеров и других требований, имеющихся в стандарте.

\section{Технология маркировки}

Практическая деятельность по маркировке туристских маршрутов включает в себя следующие (проводящиеся на местности а
соответствии с принятыми на предварительном этапе решениями и планом маркировки) виды работ и технологических операций:

\begin{itemize}
	\item рекогносцировочная разметка маршрутов,
	\item установка искусственных опор.
	\item подготовка опор к нанесению знаков маркировки,
	\item нанесение знаков маркировки,
\end{itemize}

Рекогносцировочная разметка имеет целью выбор конкретных местных предметов на трассе путешествия для последующего
нанесения на них маркировочных знаков, а также уточнение вида знаков (кроме цвета и вида марок, которые должны быть
оговорены в плане маркировки).

Рекогносцировка включает определение высоты знаков над землей, их видимости с наиболее вероятных направлений движения
туристов, согласование в случае необходимости установки туристских марок на жилых домах, строениях, межевых столбах,
иных сооружениях с отдельными лицами и организациями --- владельцами этих строений 'и сооружений,

Итогом рекогносцировки должна быть разметка мелом (карандашом) выбранных опор с указанием того, какой именно знак
надлежит здесь установить. Разметку желательно дублировать временными марками: кусками цветной бумаги или клейкой
бумажной лентой. При наличии крупномасштабной карты (схемы) на ней отмечают местонахождение будущих знаков.

Подготовка опор к нанесению маркировочных знаков состоит из обработки поверхности опор, очистки деревьев от отдельных
низкорасположенных и закрывающих вид на знаки веток и сучьев.

В местах, предназначенных для установки марок, стволы деревьев, поверхность столбов, камней, скал и стен тщательно
очищаются от грязи, мха, инородных наростов. Деревья с гладкой поверхностью (граб, бук, платан и т. п.) очищаются с
помощью тряпки. Деревья с шелушащейся и неровной корой (зрелые сосны, ель, пихта, дуб, береза и т. д.) --- с помощью
двуручного драночного ножа. Камни, скалы, стены удобно чистить металлической щеткой.

Очистку коры деревьев нужно проводить с большой осторожностью, срезая в случае необходимости лишь отдельные омертвевшие
части и не повреждая нижний влагонесущий слой.

Основной вид маркировочных работ --- нанесение знаков маркировки на местные предметы, как правило, проводится с
применением красок.

Сперва наносится фон знака белой краской, затем, поверх --- цветные элементы знака. В случае светлой поверхности
допускается нанесение цветных элементов непосредственно на предмет, а фон делать не сплошным.

Для ускорения работы целесообразно иметь специальные трафареты (шаблоны) на все
применяющиеся форматы знаков.

Для ускорения маркировочных работ на охраняемых территориях, а также при сезонном характере действия маршрутов или когда
нанесение маршрутных марок краской на стволах деревьев нежелательно по иным причинам (на ценные породы деревьев, в
заповедных хозяйствах и т. п.), возможно использование для маркировки кольцующих лент-бандажей.

Возможно применение для маркировки самоклеящейся плёнки, в том числе световозвращающей, однако надо учитывать характер
покрытия и качество сцепления с ним клеевого слоя плёнки. Так, рыхлая кора деревьев не сможет обеспечить должный
уровень адгезии, а вот металлические опоры, ограждения --- смогут.

\section{Открытые для обсуждения вопросы}
\begin{enumerate}
	\item Чем отмечать начало и конец маршрута?
	\item Двусторонний маршрут --- должны ли метки идти всегда справа, или же при движении <<по шерсти>> --- справа, в обратном направлении --- слева?
	\item Маркировка ответвлений --- в оригинальной методичке предлагалось использовать белый цвет и цвет основного маршрута, здесь заменено на серый (белый сливается с фоном). Пригодна ли такая схема? Насколько приемлемо использование трёх цветов? Может, выбрать какой-то иной формат марки для ответвлений (например, диагональная полоса)?
	\item Информация о местах отдыха, достопримечательностях, иных особенных местах --- достаточно ли СТБ, или же нужно разработать свои знаки, пригодные для нанесения краской на местности? Для примеров см. \cite{metodussr} и \cite{wikimedia-marks} (фото из Чехии).
	\item То же для указателей направлений и расстояний.
	\item Обкатать технологию и поправить соответствующий раздел: расход краски, нужные инструменты, технология нанесения.
\end{enumerate}

\begin{thebibliography}{99}
	\bibitem{metodussr} Маркировка туристских маршрутов. Методические рекомендации. Центральный совет по туризму и
		экскурсиям. Центральное рекламно-информационное бюро <<Турист>>. Москва, 1975.
	\bibitem{stb1821} СТБ~1821--2007 «Знаки информационные туристические. Общие технические условия».
	\bibitem{intrznak} Polskie Towarzystwo Turystyczno-Krajoznawcze. Instrukcja znakowania szlaków turystycznych. Warszawa, 2007.
	\bibitem{lv} Juris Smaļinskis, Aiva Jakovela. Vadlīnijas aktīvā tūrisma maršrutu marķējumam. LLTA “Lauku ceļotājs”, 2010.
	\bibitem{wikimedia-marks} \url{http://commons.wikimedia.org/wiki/Category:Hiking\_and\_footpath\_signs\_by\_country}

\end{thebibliography}

\end{document}
